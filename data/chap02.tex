% !TeX root = ../thuthesis-example.tex

\chapter{背景介绍}

\section{稀疏张量代数}
稀疏张量代数(Sparse Tensor Algebra)是指作用在存在大量零元的张量上的张量代数。与稠密张量代数不同,因为稀疏张量中仅有少量非零元,所以一般按照压缩格式存储,比如DCSR\cite{DCSR},CSB\cite{CSB},DIA\cite{DIA},CSF\cite{CSF}等。
稀疏张量代数有广泛应用。比如在机器学习中,图神经网络\cite{kipf2016semi, hamilton2017inductive}中核心算子为稀疏稠密矩阵乘法和采样稠密矩阵乘法,稀疏卷积神经网络中权重、激活等剪枝后也会出现稀疏张量运算\cite{liu2015sparse}。
在数据分析和高性能计算中,会运用稀疏张量低秩分解\cite{kolda2009tensor}和稀疏矩阵稠密向量乘法\cite{bell2012exposing}等稀疏张量代数。图~\ref{fig:sparse-intro}展示了稀疏矩阵的例子,以及两种压缩存储格式。
\begin{figure}
  \centering
  \includegraphics[width=0.99\linewidth]{稀疏矩阵示意图.pdf}
  \caption*{(a)稀疏矩阵;(b)稀疏矩阵压缩稀疏行(CSR)存储格式,由三个数组组成。行范围数组中第$i+1$个元素和第$i$个元素的差值代表第$i$行中非零元个数,列序号数组中第$i$个元素代表第$i$个非零元的列序号,非零值数组中第$i$个元素代表第$i$个非零元的值;
  (c)稀疏矩阵压缩坐标(COO)存储格式,由三个数组组成。行序号数组中第$i$个元素代表第$i$个非零元的行序号,列序号数组中第$i$个元素代表第$i$个非零元的列序号,非零值数组中第$i$个元素代表第$i$个非零元的值。}
  \caption{稀疏矩阵示例}
  \label{fig:sparse-intro}
\end{figure}

\section{稀疏稠密混合代数}
\subsection{定义}
稀疏稠密混合代数是一种稀疏张量代数,它有两种等价的表示形式:一种是张量形式(Tensor formulation,简称TF)如公式\eqref{eq:algebra-view}所示,另一种是数据库形式(Database formulation,简称DF)如公式\eqref{eq:db-view}所示。
在公式\eqref{eq:algebra-view}中,$\symbf{Y}$是输出张量,$\symbf{X^j}$是稠密输入张量,$symbf{A}$是稀疏输入张量。$symbf{A}$是稀疏的意味着至少有一个维度$a_i$是以压缩格式存储的。$y_1, y_2,\cdots,y_M$,$a_1, a_2,\cdots,a_N$,$x_1^j,x_2^j,\cdots,x_{M^j}^j$是属于相同的下标变量集合。$M$是输出张量的维度,$N$是输入稀疏张量的维度,$D$是输入稠密张量的个数,$M^j$是输入稠密张量$\symbf{X^j}$的维度。
在公式\eqref{eq:db-view}中我们使用信息传播描述描述稀疏稠密混合代数。$Q,Q_0,Q_1,Q_2$是向关联数据库的查询。
我们遵循经典的逻辑-物理分离存储思想\cite{codd1970relational}。$D$是$Q$的关联数据库,
按照$id$升序存储$(id, value)$,其中$id\in \mathbb{Z}$,$value \in \mathbb{R}^n$。$dst\in K$是任何可以哈希的键值,$f$是$K\rightarrow \mathbb{Z}$的映射。
$Q(k)$的值定义为$Q(dst)=D(f(dst))$。$\oplus$可以是任何满足交换律的算符,$\otimes$可以是任何可以接受两个对象作为输入,输出一个可被$\oplus$运算的对象。
$\oplus$的结果会被写入$Q$中的$f(dst)$位置。在DF视角下,稀疏稠密混合代数的稀疏性体现在对于所有$dst$,$Q_0(dst)$是分散的。换句话说,$Q_0(i) \bigcap Q_0(i+1) \sim \O$。
该代数的稠密性体现在$D,D_1,D_2$中的值可以是标量,稠密向量或稠密矩阵。
\begin{equation}
  \symbf{Y}_{y_1, y_2,\cdots,y_M} = \symbf{A}_{a_1, a_2,\cdots,a_N}\prod_{i=1}^{D}\symbf{X}^{j}_{x_1^j,x_2^j,\cdot,x_{M^j}^j}
  \label{eq:algebra-view}
\end{equation}
\begin{equation}
  Q(dst)=\oplus_{src\in Q_0(dst)}\{src, \otimes(Q_1(src,dst), Q_2(dst))\}
  \label{eq:db-view}
\end{equation}
在TF下,稀疏稠密混合代数特征是输入既有稀疏又有稠密张量。例如,矩阵化张量乘Khatri-Rao积MTTKRP(Matricized Tensor Times Khatri Rao Product\cite{nisa2019mttkrp},采样稠密矩阵乘法SDDMM(Sampled Dense-Dense Matrix Multiplication)\cite{yu2021exploiting},
稀疏稠密矩阵乘法SpMM(Sparse Matrix-Matrix Multiplication)\cite{huang2020ge}和张量矩阵乘TTM(Tensor Times Matrix Product)\cite{kurt2022ttm}。在TF下,MTTKRP,TTM,SDDMM和SpMM四类算子可以表示为公式\eqref{eq:four-expressions}
\begin{subequations}
  \begin{equation}
      \symbf{Y}_{i,j} = \symbf{A}_{i,k,l}\symbf{X}_{k,j}^1\symbf{X}_{l,j}^2
  \end{equation}
  \begin{equation}
      \symbf{Y}_{i,j,l} = \symbf{A}_{i,j,k}\symbf{X}_{k,l}^1
  \end{equation}
  \begin{equation}
      \symbf{Y}_{i,k} = \symbf{A}_{i,k}\symbf{X}_{i,j}^1\symbf{X}_{j,k}^2
  \end{equation}
  \begin{equation}
      \symbf{Y}_{i,k} = \symbf{A}_{i,j}\symbf{X}_{j,k}^1
  \end{equation}
  \label{eq:four-expressions}
\end{subequations}
\subsection{规约}
稀疏稠密混合代数的核心操作是规约。这一个核心观察启发我们针对规约做优化,因为我们只需要加速这一个核心操作,然后通过编译器技术来自动加速不同的稀疏稠密混合代数。
例如在TF视角下,公式\eqref{eq:algebra-view}算子在MTTKRP的$l,k$维度做规约,TTM在$k$维度,SDDMM在$j$维度,SpMM在$j$维度做规约。规约可以再一个稀疏和一个稠密维度之间进行,比如MTTKRP,TTM和SpMM。
规约也可以在两个稠密维度做,比如SDDMM。图~\ref{fig:kernels}展示了这四种算子的规约维度。我们也在图~\ref{fig:four-code}中给出了规约的具体代码例子。例如,MTTKRP包含两个规约,每个规约都和SpMM中的
规约动作一致。这样的性质也可以在DF视角下观察到。 如图~\ref{fig:redb-spmm}和图~\ref{fig:redb-mttkrp},对于MTTKRP和SpMM的第一个规约,$D_1$的值都是标量,$D_2$的值都是向量。对于MTTKRP的第二个规约,尽管$D_1$的值是一个向量,这一点和SpMM不同,
但是$\oplus$行为一致,因为$\otimes$执行的是向量的逐元素乘法。
\begin{figure}[h]%
  \centering
  \includegraphics[width=0.99\textwidth]{kernels.pdf}
  \caption{TF视角下稀疏稠密混合代数示例,连续的灰色平行四边形或正方形代表规约维度}
  \label{fig:kernels}
\end{figure}
\begin{figure}[h]%
  \centering
  \includegraphics[width=0.99\textwidth]{SpHY.pdf}
  \caption{TF视角下稀疏稠密混合代数规约的代码示例。黄色的和绿色的代码行是规约代码。MTTKRP有两个层次的归于,分别用黄色和绿色表示。重叠部分代表第一个层级的规约输出是第二个层级规约的输入。对于A的存储我们遵循\cite{kjolstad:2020:phd-thesis}的命名规则}
  \label{fig:four-code}
\end{figure}
\begin{figure}[h]%
  \centering
  \includegraphics[width=0.99\textwidth]{reduction-db-spmm.pdf}
  \caption{SpMM规约操作示意图,下方展示了该算子在TF和DF视角下等效的表达}
  \label{fig:redb-spmm}
\end{figure}
\begin{figure}[h]%
  \centering
  \includegraphics[width=0.99\textwidth]{reduction-db-mttkrp.pdf}
  \caption{MTTKRP规约操作示意图,下方展示了该算子在TF和DF视角下等效的表达}
  \label{fig:redb-mttkrp}
\end{figure}

\section{稀疏稠密张量代数GPU优化技术}
如上节所述,规约是稀疏稠密混合张量代数的核心操作,不同算子之间会共享同类型的规约。因此,在

\section{插图}

图片通常在 \env{figure} 环境中使用 \cs{includegraphics} 插入,如图~\ref{fig:example} 的源代码。
建议矢量图片使用 PDF 格式,比如数据可视化的绘图;
照片应使用 JPG 格式;
其他的栅格图应使用无损的 PNG 格式。
注意,LaTeX 不支持 TIFF 格式;EPS 格式已经过时。

\begin{figure}
  \centering
  \includegraphics[width=0.5\linewidth]{example-image-a.pdf}
  \caption*{国外的期刊习惯将图表的标题和说明文字写成一段,需要改写为标题只含图表的名称,其他说明文字以注释方式写在图表下方,或者写在正文中。}
  \caption{示例图片标题}
  \label{fig:example}
\end{figure}

若图或表中有附注,采用英文小写字母顺序编号,附注写在图或表的下方。
国外的期刊习惯将图表的标题和说明文字写成一段,需要改写为标题只含图表的名称,其他说明文字以注释方式写在图表下方,或者写在正文中。

如果一个图由两个或两个以上分图组成时,各分图分别以 (a)、(b)、(c)...... 作为图序,并须有分图题。
推荐使用 \pkg{subcaption} 宏包来处理, 比如图~\ref{fig:subfig-a} 和图~\ref{fig:subfig-b}。

\begin{figure}
  \centering
  \subcaptionbox{分图 A\label{fig:subfig-a}}
    {\includegraphics[width=0.35\linewidth]{example-image-a.pdf}}
  \subcaptionbox{分图 B\label{fig:subfig-b}}
    {\includegraphics[width=0.35\linewidth]{example-image-b.pdf}}
  \caption{多个分图的示例}
  \label{fig:multi-image}
\end{figure}



\section{表格}

表应具有自明性。为使表格简洁易读,尽可能采用三线表,如表~\ref{tab:three-line}。
三条线可以使用 \pkg{booktabs} 宏包提供的命令生成。

\begin{table}
  \centering
  \caption{三线表示例}
  \begin{tabular}{ll}
    \toprule
    文件名          & 描述                         \\
    \midrule
    thuthesis.dtx   & 模板的源文件,包括文档和注释 \\
    thuthesis.cls   & 模板文件                     \\
    thuthesis-*.bst & BibTeX 参考文献表样式文件    \\
    \bottomrule
  \end{tabular}
  \label{tab:three-line}
\end{table}

表格如果有附注,尤其是需要在表格中进行标注时,可以使用 \pkg{threeparttable} 宏包。
研究生要求使用英文小写字母 a、b、c……顺序编号,本科生使用圈码 ①、②、③……编号。

\begin{table}
  \centering
  \begin{threeparttable}[c]
    \caption{带附注的表格示例}
    \label{tab:three-part-table}
    \begin{tabular}{ll}
      \toprule
      文件名                 & 描述                         \\
      \midrule
      thuthesis.dtx\tnote{a} & 模板的源文件,包括文档和注释 \\
      thuthesis.cls\tnote{b} & 模板文件                     \\
      thuthesis-*.bst        & BibTeX 参考文献表样式文件    \\
      \bottomrule
    \end{tabular}
    \begin{tablenotes}
      \item [a] 可以通过 xelatex 编译生成模板的使用说明文档;
        使用 xetex 编译 \file{thuthesis.ins} 时则会从 \file{.dtx} 中去除掉文档和注释,得到精简的 \file{.cls} 文件。
      \item [b] 更新模板时,一定要记得编译生成 \file{.cls} 文件,否则编译论文时载入的依然是旧版的模板。
    \end{tablenotes}
  \end{threeparttable}
\end{table}

如某个表需要转页接排,可以使用 \pkg{longtable} 宏包,需要在随后的各页上重复表的编号。
编号后跟表题(可省略)和“(续)”,置于表上方。续表均应重复表头。

\begin{longtable}{cccc}
    \caption{跨页长表格的表题}
    \label{tab:longtable} \\
    \toprule
    表头 1 & 表头 2 & 表头 3 & 表头 4 \\
    \midrule
  \endfirsthead
    \caption*{续表~\thetable\quad 跨页长表格的表题} \\
    \toprule
    表头 1 & 表头 2 & 表头 3 & 表头 4 \\
    \midrule
  \endhead
    \bottomrule
  \endfoot
  Row 1  & & & \\
  Row 2  & & & \\
  Row 3  & & & \\
  Row 4  & & & \\
  Row 5  & & & \\
  Row 6  & & & \\
  Row 7  & & & \\
  Row 8  & & & \\
  Row 9  & & & \\
  Row 10 & & & \\
\end{longtable}



\section{算法}

算法环境可以使用 \pkg{algorithms} 或者 \pkg{algorithm2e} 宏包。

\renewcommand{\algorithmicrequire}{\textbf{输入:}\unskip}
\renewcommand{\algorithmicensure}{\textbf{输出:}\unskip}

\begin{algorithm}
  \caption{Calculate $y = x^n$}
  \label{alg1}
  \small
  \begin{algorithmic}
    \REQUIRE $n \geq 0$
    \ENSURE $y = x^n$

    \STATE $y \leftarrow 1$
    \STATE $X \leftarrow x$
    \STATE $N \leftarrow n$

    \WHILE{$N \neq 0$}
      \IF{$N$ is even}
        \STATE $X \leftarrow X \times X$
        \STATE $N \leftarrow N / 2$
      \ELSE[$N$ is odd]
        \STATE $y \leftarrow y \times X$
        \STATE $N \leftarrow N - 1$
      \ENDIF
    \ENDWHILE
  \end{algorithmic}
\end{algorithm}
