% !TeX root = ../thuthesis-example.tex

\chapter{主要挑战和核心贡献}
本章总结了稀疏算子编译器设计在优化空间扩展和编译算法设计两方面面临的主要挑战,同时介绍了本文对于解决这些挑战的核心贡献。
\section{优化空间扩展}
\subsection{主要挑战}
虽然GPU提供了10TFLOP/s量级的并行计算能力,但是由于稀疏张量计算的不规则性,充分利用GPU的并行计算能力仍具有挑战性。表\ref{tab:motivation-1}展示了SDDMM\cite{yu2021exploiting},SpMM\cite{huang2020ge},MTTKRP\cite{nisa2019mttkrp},其中每种算子均采用当前性能最高的开源库。
\begin{table}
  \centering
  \caption{稀疏张量算子库性能和GPU峰值算力对比}
  \begin{tabular}{llll}
    \toprule
    算子名  & 算子性能(TFLOP/s) & 平台峰值算力(TFLOP/s) & 与峰值算力差距倍数   \\
    \midrule
    SDDMM  & 0.42 & 15.7 & 37.4x \\
    SpMM   & 0.28 & 10.6 & 37.9x \\
    MTTKRP & 0.20 & 15.7 & 78.5x \\
    \bottomrule
  \end{tabular}
  \label{tab:motivation-1}
\end{table}
从表\ref{tab:motivation-1}中可以看出,最佳的开源稀疏张量算子库算子性能和平台峰值算力还有几十倍的差距,这说明计算平台的算力还没有被充分利用。因此主要挑战是如何进一步扩展优化空间,即写出更高效的稀疏张量算子。
\subsection{核心贡献}
本文提出了灵活规约,这是一种新的针对SIMT架构上稀疏稠密混合张量代数算子优化技术。该技术扩展了优化空间,进一步提升了算子库性能。实验表明,采用灵活规约同步后可以将SpMM算子库性能提升平均1.6至2.3倍(针对不同代GPU架构加速比有所不同)。
\section{编译算法设计}
除了算子性能较低,现在编写高性能稀疏算子库也较为困难。这里用代码行数来衡量编写算子库的难易程度。表\ref{tab:motivation-2}展示了SDDMM,SpMM,MTTKRP最佳算子库和稀疏张量编译器TACO的对比。其中算子库代码行数指编写GPU上执行的CUDA算子需要的代码行数(不包括CPU和GPU间数据搬移和算子调用的代码),算子编译器代码行数指运用TACO提供的调度变换指令编写算子所需要的代码行数。
\begin{table}
  \centering
  \begin{threeparttable}[c]
  \caption{稀疏张量算子库和算子编译器的性能和代码行数对比}
  \label{tab:motivation-2}
  \begin{tabular}{lllll}
    \toprule
    算子名  & 算子库 & 算子编译器 & 算子库与编译器 & 算子库与编译器   \\
           & 代码行数 & 代码行数 & 代码行数比    & 算子性能比       \\
    \midrule
    SDDMM\tnote{a}  & 53 & 12 & 4.4x & 2.1x \\
    SpMM\tnote{b}   & 132 & 13 & 10.2x & 2.6x \\
    MTTKRP\tnote{c} & 40 & 12 & 3.3x & 1.2x \\
    \bottomrule
  \end{tabular}
  \begin{tablenotes}
    \item [a] 选取SOTA的GPU开源SDDMM算子库 PRedS \cite{yu2021exploiting}中的sddmm\_csr\_ebalance\_vec4函数和TACO中的scheduleSDDMMGPU函数做对比。
    \item [b] 选取SOTA的GPU开源SpMM算子库 DASpMM \cite{dai2022heuristic}中的采用的四种算子:csrspmm\_rowcaching\_rowbalance\_kernel,
      csrspmm\_rowcaching\_nnzbalance\_kernel,csrspmm\_parreduce\_rowbalance\_kernel和csrspmm\_parreduce\_nnzbalance\_kernel函数的平均行数和TACO中的scheduleSpMMGPU函数对比。
    \item [c] 选取SOTA的GPU开源MTTKRP算子库MM-CSF\cite{nisa2019mttkrp}中采用HyB格式的3维MTTKRP函数的平均行数和TACO中的scheduleMTTKRPGPU函数对比。
  \end{tablenotes}
  \end{threeparttable}
\end{table}
从表\ref{tab:motivation-2}中可以看出,利用编译器提供的领域专用语言可以减小3到10倍的代码量,但是会牺牲16\%到60\%的性能。性能下降说明编译器表达的优化空间没有包含已有算子库的优化技巧。
因此,需要设计更好的编译算法,在进一步降低用户代码量的同时,扩展优化空间,从而使得用户可以用更低的开发难度得到更高性能的稀疏张量算子。
\subsection{核心贡献}
基于灵活规约,本文提出了细分线程组,这是一种新的针对SIMT架构上稀疏稠密混合张量代数的编译算法。该技术扩展了稀疏算子编译器表达的优化空间,同时在用户端只需要增加一行代码即可获得灵活规约带来的算子加速。
实验表明,采用细分线程组后,编译器生成SpMM算子性能提升了平均1.2倍,最多提升3.8倍;编译器生成MTTKRP算子性能最多提升2.7倍。

\section{数学符号}

中文论文的数学符号默认遵循 GB/T 3102.11—1993《物理科学和技术中使用的数学符号》
\footnote{原 GB 3102.11—1993,自 2017 年 3 月 23 日起,该标准转为推荐性标准。}。
该标准参照采纳 ISO 31-11:1992 \footnote{目前已更新为 ISO 80000-2:2019。},
但是与 \TeX{} 默认的美国数学学会(AMS)的符号习惯有所区别。
具体地来说主要有以下差异:
\begin{enumerate}
  \item 大写希腊字母默认为斜体,如
    \begin{equation*}
      \Gamma \Delta \Theta \Lambda \Xi \Pi \Sigma \Upsilon \Phi \Psi \Omega.
    \end{equation*}
    注意有限增量符号 $\increment$ 固定使用正体,模板提供了 \cs{increment} 命令。
  \item 小于等于号和大于等于号使用倾斜的字形 $\le$、$\ge$。
  \item 积分号使用正体,比如 $\int$、$\oint$。
  \item
    偏微分符号 $\partial$ 使用正体。
  \item
    省略号 \cs{dots} 按照中文的习惯固定居中,比如
    \begin{equation*}
      1, 2, \dots, n \quad 1 + 2 + \dots + n.
    \end{equation*}
  \item
    实部 $\Re$ 和虚部 $\Im$ 的字体使用罗马体。
\end{enumerate}

以上数学符号样式的差异可以在模板中统一设置。
另外国标还有一些与 AMS 不同的符号使用习惯,需要用户在写作时进行处理:
\begin{enumerate}
  \item 数学常数和特殊函数名用正体,如
    \begin{equation*}
      \uppi = 3.14\dots; \quad
      \symup{i}^2 = -1; \quad
      \symup{e} = \lim_{n \to \infty} \left( 1 + \frac{1}{n} \right)^n.
    \end{equation*}
  \item 微分号使用正体,比如 $\dif y / \dif x$。
  \item 向量、矩阵和张量用粗斜体(\cs{symbf}),如 $\symbf{x}$、$\symbf{\Sigma}$、$\symbfsf{T}$。
  \item 自然对数用 $\ln x$ 不用 $\log x$。
\end{enumerate}


英文论文的数学符号使用 \TeX{} 默认的样式。
如果有必要,也可以通过设置 \verb|math-style| 选择数学符号样式。

关于量和单位推荐使用
\href{http://mirrors.ctan.org/macros/latex/contrib/siunitx/siunitx.pdf}{\pkg{siunitx}}
宏包,
可以方便地处理希腊字母以及数字与单位之间的空白,
比如:
\SI{6.4e6}{m},
\SI{9}{\micro\meter},
\si{kg.m.s^{-1}},
\SIrange{10}{20}{\degreeCelsius}。



\section{数学公式}

数学公式可以使用 \env{equation} 和 \env{equation*} 环境。
注意数学公式的引用应前后带括号,通常使用 \cs{eqref} 命令,比如式\eqref{eq:example}。
\begin{equation}
  \frac{1}{2 \uppi \symup{i}} \int_\gamma f = \sum_{k=1}^m n(\gamma; a_k) \mathscr{R}(f; a_k).
  \label{eq:example}
\end{equation}

多行公式尽可能在“=”处对齐,推荐使用 \env{align} 环境。
\begin{align}
  a & = b + c + d + e \\
    & = f + g
\end{align}



\section{数学定理}

定理环境的格式可以使用 \pkg{amsthm} 或者 \pkg{ntheorem} 宏包配置。
用户在导言区载入这两者之一后,模板会自动配置 \env{thoerem}、\env{proof} 等环境。

\begin{theorem}[Lindeberg--Lévy 中心极限定理]
  设随机变量 $X_1, X_2, \dots, X_n$ 独立同分布, 且具有期望 $\mu$ 和有限的方差 $\sigma^2 \ne 0$,
  记 $\bar{X}_n = \frac{1}{n} \sum_{i+1}^n X_i$,则
  \begin{equation}
    \lim_{n \to \infty} P \left(\frac{\sqrt{n} \left( \bar{X}_n - \mu \right)}{\sigma} \le z \right) = \Phi(z),
  \end{equation}
  其中 $\Phi(z)$ 是标准正态分布的分布函数。
\end{theorem}
\begin{proof}
  Trivial.
\end{proof}

同时模板还提供了 \env{assumption}、\env{definition}、\env{proposition}、
\env{lemma}、\env{theorem}、\env{axiom}、\env{corollary}、\env{exercise}、
\env{example}、\env{remar}、\env{problem}、\env{conjecture} 这些相关的环境。
